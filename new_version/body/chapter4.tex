\section{已知问题和未来发展}
\subsection{已知问题}
\sout{本模板未采用2013版规范的页边距设置,因为实在是办不到2.5CM顶部页边距加上2.6CM的页眉设置啊。}

本模板办到了页眉为 2.6cm,页脚为 2.4cm的武理(无理)要求,就是能想到的实现办法需要在目录后增加一个空白页,这样后面的页面设置就正常了。由于Latex和Word的计算间距方式不同,本模板只实现了神似,并没有完全参透两者的转换关系。

latex的行距计算方法不同,因此段后距本模板会大于0.5行,实际使用并不会很明显。

\subsection{未来发展}
武汉理工大学本科生论文的未来发展还是需要各位用户的参与。
本模板还是依赖于CTEX,而CTEX已经有些过时了。希望后来人能够更加强大吧!奥利给!
希望在校的对Latex有兴趣的能够聚集起来,个人对模板的优化有限,团队的力量才是无限的。

\subsection{官方认证}
到目前为止(\today )没有武汉理工大学任何官方组织对于本模板的格式或者内容进行认证,这代表采用本模板进行的论文写作可能不被官方的论文系统接受。如在进行原创性(防抄袭)检测的时候,可能需要提供提供doc版本的论文。希望用户了解到这个潜在的风险,做好文件转换和备份的准备。本人不对任何由于使用本模板而导致的毕业论文纠纷承担任何责任!