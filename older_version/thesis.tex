\documentclass[a4paper,zihao = -4]{ctexart}

\usepackage{amsmath}                % AMS LaTeX宏包
\usepackage[style=1]{mdframed}
\usepackage{amsthm}
\usepackage{amsfonts}

\usepackage{xeCJK}
\newcommand{\zhongsong}{\CJKfontspec{STZhongsong}}	%华文中宋
\usepackage{setspace}				%设置间距
\usepackage{ulem}

\usepackage{calc}	%包含textwidth,textheight
\usepackage{geometry}               % 页边距调整
\geometry{top=2.5cm,bottom=2cm,left=2.5cm,right=2cm}

\usepackage{mathrsfs}                % 英文花体字 体
\usepackage{bm}                      % 数学公式中的黑斜体
\usepackage{bbding,manfnt}           % 一些图标,如 \dbend
\usepackage{lettrine}                % 首字下沉,命令\lettrine
\def\attention{\lettrine[lines=2,lraise=0,nindent=0em]{\large\textdbend\hspace{1mm}}{}}
\usepackage{longtable}
\usepackage[toc,page]{appendix}

%====================公式按章编号==========================
\numberwithin{equation}{section}
\numberwithin{table}{section}
\numberwithin{figure}{section}
%================== 图形支持宏包 =========================
\usepackage{subfigure}
\usepackage{graphicx}                % 嵌入png图像
\usepackage{color,xcolor}            % 支持彩色文本、底色、文本框等
\usepackage{hyperref}                % 交叉引用
\usepackage{caption}
\captionsetup{figurewithin=section}
%==================== 源码和流程图 =====================
\usepackage{listings}                % 粘贴源代码
\usepackage{tikz}                    
\usepackage{tikz-3dplot}
\usetikzlibrary{shapes,arrows,positioning}

%================= 基本格式预置 ===========================
\usepackage{fancyhdr}
\setmainfont{Times New Roman}

\ctexset{
	section = {
		format = \centering\bfseries\zihao{-2} \heiti,
		name = {第, 章}
	},
	subsection = {
		nameformat = \bfseries\zihao{3} \heiti
	},
	subsubsection = {
		nameformat = \bfseries\zihao{4} \heiti
	}
}

\begin{document}

%===================  定理类环境定义 ===================
\newtheorem{example}{例}              % 整体编号
\newtheorem{algorithm}{算法}
\newtheorem{theorem}{定理}            % 按 section 编号
\newtheorem{definition}{定义}
\newtheorem{axiom}{公理}
\newtheorem{property}{性质}
\newtheorem{proposition}{命题}
\newtheorem{lemma}{引理}
\newtheorem{corollary}{推论}
\newtheorem{remark}{注解}
\newtheorem{condition}{条件}
\newtheorem{conclusion}{结论}
\newtheorem{assumption}{假设}
%==================重定义 ===================
\renewcommand{\contentsname}{目 ~~ 录}     
\renewcommand{\abstractname}{摘 ~~ 要} 
\renewcommand{\refname}{参考文献}     
\renewcommand{\indexname}{索引}
\renewcommand{\figurename}{图}
\renewcommand{\tablename}{表}
\renewcommand{\appendixname}{附录}
\renewcommand{\proofname}{证明}
\renewcommand{\algorithm}{算法} 

%============== 封面、声明授权和摘要 =================
\pagestyle{plain}
%===============  封面  =================
\smallskip
\begin{center}


\vspace*{1.0cm}
{\zhongsong \zihao{1} 武汉理工大学毕业设计(论文)}\\
\vspace*{4.0cm}
 \heiti{\zihao{2} LSTM循环神经网络的时间序列预测研究 }\\
\vspace*{4.0cm}
\songti
\begin{tabular}{cc}
 \zhongsong \zihao{3} 
 学院(系):&\underline{\makebox[7cm][c]{\zihao{-2}理学院}} \\ 
 \\
 \zhongsong \zihao{3}
 专业班级: & \underline{\makebox[7cm][c]{\zihao{-2}信计1602班}} \\ 
 \\
  \zhongsong \zihao{3}
 学生姓名: & \underline{\makebox[7cm][c]{\zihao{-2}顾焕申}} \\ 
 \\
  \zhongsong \zihao{3}
  指导教师: & \underline{\makebox[7cm][c]{\zihao{-2}万源}} \\ 
 \\
\end{tabular} 
\end{center}
\thispagestyle{empty}
\clearpage
%=====================原创性声明===========
\begin{spacing}{2.0}
\begin{center}
\heiti \zihao{-2} \textbf{学位论文原创性声明}
\end{center}

\zihao{-4}
本人郑重声明:所呈交的论文是本人在导师的指导下独立进行研究所取得的研究成果。除了文中特别加以标注引用的内容外,本论文不包括任何其他个人或集体已经发表或撰写的成果作品。本人完全意识到本声明的法律后果由本人承担。 
\begin{flushright}
作者签名:\qquad\qquad\qquad\qquad \\

年\qquad 月\qquad 日
\end{flushright}
\vskip 2cm
\begin{center}
\heiti \zihao{-2} \textbf{学位论文版权使用授权书}
\end{center}

\zihao{-4}
本学位论文作者完全了解学校有关保障、使用学位论文的规定,同意学校保留并向有关学位论文管理部门或机构送交论文的复印件和电子版,允许论文被查阅和借阅。本人授权省级优秀学士论文评选机构将本学位论文的全部或部分内容编入有关数据进行检索,可以采用影印、缩印或扫描等复制手段保存和汇编本学位论文。
\smallskip

本学位论文属于
\begin{tabular}[t]{l}
1、保密$ \Box$,在~~~~年解密后适用本授权书  \\ 
2、不保密$ \Box$  \\ 
\end{tabular} \\
\begin{center}
(请在以上相应方框内打“$\surd”$)
\end{center}
\begin{flushright}
\zihao{-4} 作者签名:  \quad\quad\quad\quad\quad 年 \quad  月  \quad  日\\
导师签名:   \quad\quad\quad\quad\quad 年 \quad  月 \quad   日\\
\end{flushright}
\end{spacing}
\thispagestyle{empty}
\clearpage
\pagenumbering{Roman}
\section*{\heiti \zihao{-2} \centering 摘 ~~ 要}

\vskip0.5cm
\zihao{-4}
本文借助计算流体力学软件 FLUENT 首先针对一日产 650 吨的空气燃烧的燃油浮 法 玻 璃 熔 窑 火 焰 空 间 进 行 了 三 维 数 值 模 拟 ,×××××××××××××××××××××××××××××××××××对两种情况进行了比较,所得结果对于×××具有重要的指导意义。 

论文主要研究了××××××××××××××××××××××。 

研 究 结 果 表 明 : ××××××××××××××××××××××××××××××。 

本文的特色:××××××××××××××××××××××××。

\textbf{\heiti \zihao{4} 关键词:}  动力定位, 船舶操纵性 ,控制方法,状态估计算法
\addcontentsline{toc}{section}{摘要}

\clearpage


\section*{\zihao{-2} \centering \textbf{Abstract} }

\zihao{-4}
This paper first simulates the combustion space of a 650t/day air-fuel combustion 
float glass furnace.Then transform it into a oxy-fuel one with the model and compare 
them. The results have important guiding significance in transforming float glass furnace 
from air-fuel to oxy-fuel combustion.   

××××××××××××××××××××××××××××××

××××××××××××××××××××××××××××××

\textbf{\zihao{4} Key Words:} Dynamic Positioning, Ship Manoeuvrability ,Control Algorithm, State Estimate Algorithm
\addcontentsline{toc}{section}{Abstract}


\pagestyle{empty}
\tableofcontents 
\thispagestyle{empty}

%============== 论文正文   =================
\pagestyle{fancy}
\fancyhf{} 
\setlength{\headsep}{20pt}
\setlength{\headheight}{0cm}
\setlength{\textwidth}{\paperwidth - 4.5cm}
\setlength{\textheight}{\paperheight - 6.08cm - \headsep - \footskip}
\setlength\topmargin{0.17cm}
\fancyhead[C]{\zihao{5}  \songti 武汉理工大学毕业设计(论文)\\}
\fancyfoot[C]{\zihao{5} \thepage}
\renewcommand{\headrulewidth}{0.65pt} 

\newpage
如你所见,本页为了解决这个第一行文字与页眉分割线重合问题,我们添加了一个空白页。
\vspace*{\fill}
\begin{center}   
	{\color{red}\fontsize{50} 1此页记得删除}
\end{center}
\vspace*{\fill}
\newpage

\setlength{\lineskip}{20pt} %设置行间距
\setlength{\parskip}{0.5\baselineskip - 10pt} %设置行间距

\include{body/chapter1} 
\include{body/chapter2}
\section{进阶功能}
\subsection{文献管理}
文献管理经过更新后通过自己手动来输入,主要通过百度学术上的引用。正文中效果如本处\upcite{zhongwen1}。

\section{已知问题和未来发展}
\subsection{已知问题}
\sout{本模板未采用2013版规范的页边距设置,因为实在是办不到2.5CM顶部页边距加上2.6CM的页眉设置啊。}

本模板办到了页眉为 2.6cm,页脚为 2.4cm的武理(无理)要求,就是能想到的实现办法需要在目录后增加一个空白页,这样后面的页面设置就正常了。由于Latex和Word的计算间距方式不同,本模板只实现了神似,并没有完全参透两者的转换关系。

latex的行距计算方法不同,因此段后距本模板会大于0.5行,实际使用并不会很明显。

\subsection{未来发展}
武汉理工大学本科生论文的未来发展还是需要各位用户的参与。
本模板还是依赖于CTEX,而CTEX已经有些过时了。希望后来人能够更加强大吧!奥利给!
希望在校的对Latex有兴趣的能够聚集起来,个人对模板的优化有限,团队的力量才是无限的。

\subsection{官方认证}
到目前为止(\today )没有武汉理工大学任何官方组织对于本模板的格式或者内容进行认证,这代表采用本模板进行的论文写作可能不被官方的论文系统接受。如在进行原创性(防抄袭)检测的时候,可能需要提供提供doc版本的论文。希望用户了解到这个潜在的风险,做好文件转换和备份的准备。本人不对任何由于使用本模板而导致的毕业论文纠纷承担任何责任!
%============= 参考文献 =====================
\bibliographystyle{gbt7714-2005}     %论文引用格式
\addcontentsline{toc}{section}{参考文献}
\bibliography{bibfile}
\clearpage
\newpage
\appendix
\begin{thebibliography}{99}
\songti\zihao{5}
\bibitem{zhongwen1}刘颉羲, 陈松灿. 基于混合门单元的非平稳时间序列预测[J]. 计算机研究与发展, 2019, 56(8): 1642 – 1651.
\bibitem{Guo-4}Guo J, Xie Z, Qin Y, et al. Short-Term Abnormal Passenger Flow Prediction Based on the Fusion of SVR and LSTM[J]. IEEE Access, 2019, 7 : 42946–42955. 

\end{thebibliography}
%=============  致谢  ======================
\section*{致谢}
\addcontentsline{toc}{section}{致谢}
(原致谢)感谢父母为我提供的良好的衣食条件,让我有精力投入到这项没有经济回报的项目中去。
感谢徐海祥老师为我定制的论文题目,这个题目让我有兴趣制作这个模板。感谢武汉理工大学博士与硕士论文作者Hu,Weiyi,我在本模板制作的过程中参考了前辈的思路的方法。我研究过的模板还包括:上海交通大学,清华大学,哈尔滨工业大学,以及中国科技大学。其中论文引用格式GBT7714-2005-BibTeX-Style是上海财经大学的Haixing Hu作品,本模板离不开这些有益的资源的支持。同样感谢正在使用这个模板的你,相信通过你们的使用和传播,这个模板会变得越来越完善。


(新致谢)感谢武汉理工大学本科毕业论文的作者Caoyu,本人对模板进行了适当修改:包括封面、页面设置等,以符合《武汉理工大学本科生毕业设计(论文)撰写规范和示例。同时也感谢Wang,Xiaoxiong同学对页眉设置的帮助。
\end{document}